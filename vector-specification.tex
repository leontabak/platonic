\documentclass[twoside]{article}
\title{Specification of Vectors and Matrices}
\author{CSC315 Programming Language Concepts}
\date{01 October 2013}
\usepackage{amsmath}
\begin{document}
\maketitle

\section{Vector2D}

\subsection{Addition}

\paragraph{Here is the general rule:}

\begin{align*}
  \vec{u} & = (u_x, u_y) \\
  \vec{v} & = (v_x, v_y) \\
  \vec{u} + \vec{v} & = (u_x + v_x, u_y + v_y) 
  \end{align*}

\paragraph{Here is a specific example:}

\begin{align*}
  \vec{u} & = (3, 4) \\
  \vec{v} & = (5, 12) \\
  \vec{u} + \vec{v} & = (3 + 5, 4 + 12) \\
                          & = (8, 16)
  \end{align*}

\subsection{Subtraction}

\paragraph{Here is the general rule:}

\begin{align*}
  \vec{u} & = (u_x, u_y) \\
  \vec{v} & = (v_x, v_y) \\
  \vec{u} + \vec{v} & = (u_x - v_x, u_y - v_y) 
  \end{align*}

\paragraph{Here is a specific example:}

\begin{align*}
  \vec{u} & = (3, 4) \\
  \vec{v} & = (5, 12) \\
  \vec{v} - \vec{u} & = (5 - 3, 12 - 4) \\
                          & = (2, 8)
  \end{align*}

\subsection{Multiplication by a scalar}

\paragraph{Here is the general rule:}

\begin{align*}
  \vec{u} & = (u_x, u_y) \\
   scaleFactor \times \vec{u} & = (scaleFactor \times u_x, scaleFactor \times u_y)
  \end{align*}

\paragraph{Here is a specific example:}

\begin{align*}
  \vec{u} & = (3, 4) \\
  2 \vec{u} & = (6, 8)
  \end{align*}

\subsection{Dot product}

\paragraph{Here is the general rule:}

\begin{align*}
  \vec{u} & = (u_x, u_y) \\
  \vec{v} & = (v_x, v_y) \\
  \vec{u} \cdot \vec{v} & = u_x v_x + u_y v_y
  \end{align*}

\paragraph{Here is a specific example:}

\begin{align*}
  \vec{u} & = (3, 4) \\
  \vec{v} & = (5, 12) \\
  \vec{u} \cdot \vec{v} & = 3 \cdot 5 + 4 \cdot 12 \\
                             & = 15 + 48 \\
                             & = 63
  \end{align*}

\begin{align*}
  \vec{u} \cdot \vec{v} & = |\vec{u}| |\vec{v}| \cos \psi \;\;\;
    \parbox{8cm}{where $\psi$ is the angle between the vectors}
  \end{align*}

\subsection{Magnitude}

\paragraph{Here is the general rule:}

\begin{align*}
  \vec{u} & = (u_x, u_y) \\
  |\vec{u}| & = \sqrt{ u_x^2 + u_y^2} \\
             & = \sqrt{ \vec{u} \cdot \vec{u} }
  \end{align*}

\paragraph{Here is a specific example:}

\begin{align*}
  \vec{u} & = (3, 4) \\
  |\vec{u}| & = \sqrt{ 3^2 + 4^2} \\
              & = \sqrt{9 + 16} \\
              & = \sqrt{ 25 } \\
              & = 5
  \end{align*}

\subsection{Normalize}

\paragraph{Here is the general rule:}

\begin{align*}
  \vec{u} & = (u_x, u_y) \\
  \hat{u} & = \frac{1}{|\vec{u}|} \vec{u} \\
            & = (\frac{u_x}{\sqrt{u_x^2 + u_y^2}}, \frac{u_y}{\sqrt{u_x^2 + u_y^2}})
  \end{align*}

\paragraph{Here is a specific example:}

\begin{align*}
  \vec{u} & = (3, 4) \\
  |\vec{u}| & = 5 \\
  \hat{u} & = \frac{1}{|\vec{u}|} \vec{u} \\
            & = (\frac{3}{5}, \frac{4}{5})
  \end{align*}

\section{Vector3D}

\paragraph{Here is the general rule:}

\subsection{Addition}

\begin{align*}
  \vec{u} & = (u_x, u_y, u_z) \\
  \vec{v} & = (v_x, v_y, v_z) \\
  \vec{u} + \vec{v} & = (u_x + v_x, u_y + v_y, u_z + v_z) 
  \end{align*}

\paragraph{Here is a specific example:}

\begin{align*}
  \vec{u} & = (1, 2, 3) \\
  \vec{v} & = (4, 5, 6) \\
  \vec{u} + \vec{v} & = (1 + 4, 2 + 5, 3 + 6) \\
                          & = (5, 7, 9)
  \end{align*}

\subsection{Subtraction}

\paragraph{Here is the general rule:}

\begin{align*}
  \vec{u} & = (u_x, u_y, u_z) \\
  \vec{v} & = (v_x, v_y, v_z) \\
  \vec{u} + \vec{v} & = (u_x - v_x, u_y - v_y, u_z - v_z) 
  \end{align*}

\paragraph{Here is a specific example:}

\begin{align*}
  \vec{u} & = (1, 2, 3) \\
  \vec{v} & = (4, 5, 6) \\
  \vec{v} - \vec{u} & = (4 - 1, 5 - 2, 6 - 3) \\
                          & = (3, 3, 3)
  \end{align*}

\subsection{Multiplication by a scalar}

\paragraph{Here is the general rule:}

\begin{align*}
  \vec{u} & = (u_x, u_y) \\
   scaleFactor \times \vec{u} & = (scaleFactor \times u_x, scaleFactor \times u_y)
  \end{align*}

\paragraph{Here is a specific example:}

\begin{align*}
  \vec{u} & = (1, 2, 3) \\
  2 \vec{u} & = (2, 4, 6)
  \end{align*}

\subsection{Dot product}

\paragraph{Here is the general rule:}

\begin{align*}
  \vec{u} & = (u_x, u_y, u_z) \\
  \vec{v} & = (v_x, v_y, v_z) \\
  \vec{u} \cdot \vec{v} & = u_x v_x + u_y v_y + u_z v_z
  \end{align*}

\paragraph{Here is a specific example:}

\begin{align*}
  \vec{u} & = (1, 2, 3) \\
  \vec{v} & = (4, 5, 6) \\
  \vec{u} \cdot \vec{v} & = 1 \cdot 4 + 2 \cdot 5 + 3 \cdot 6 \\
                             & = 4 + 10 + 18 \\
                             & = 30
  \end{align*}

\begin{align*}
  \vec{u} \cdot \vec{v} & = |\vec{u}| |\vec{v}| \cos \psi \;\;\;
    \parbox{8cm}{where $\psi$ is the angle between the vectors}
  \end{align*}

\subsection{Magnitude}

\paragraph{Here is the general rule:}

\begin{align*}
  \vec{u} & = (u_x, u_y, u_z) \\
  |\vec{u}| & = \sqrt{ u_x^2 + u_y^2 + u_z^2} \\
             & = \sqrt{ \vec{u} \cdot \vec{u} }
  \end{align*}

\paragraph{Here is a specific example:}

\begin{align*}
  \vec{u} & = (1, 2, 2) \\
  |\vec{u}| & = \sqrt{ 1^2 + 2^2 + 2^2} \\
              & = \sqrt{1 + 4 + 4} \\
              & = \sqrt{ 9 } \\
              & = 3
  \end{align*}

\subsection{Normalize}

\paragraph{Here is the general rule:}

\begin{align*}
  \vec{u} & = (u_x, u_y, u_z) \\
  \hat{u} & = \frac{1}{|\vec{u}|} \vec{u} \\
            & = (\frac{u_x}{\sqrt{u_x^2 + u_y^2 + u_z^2}}, 
                    \frac{u_y}{\sqrt{u_x^2 + u_y^2 + u_z^2}})
  \end{align*}

\paragraph{Here is a specific example:}

\begin{align*}
  \vec{u} & = (1, 2, 2) \\
  |\vec{u}| & = 3 \\
  \hat{u} & = \frac{1}{|\vec{u}|} \vec{u} \\
            & = (\frac{1}{3}, \frac{2}{3}, \frac{2}{3})
  \end{align*}

\section{Matrix2x2}

\subsection{Special $2 \times 2$ matrices}

\subsubsection{Identity}

\begin{align*}
  I & = \left[ \begin{array}{rr}
    1 & 0 \\
    0 & 1
    \end{array} \right]
  \end{align*}

\subsubsection{Rotation}

\paragraph{Here is the general rule:}

\begin{align*}
  R( \psi ) & = \left[ \begin{array}{rr}
    \cos \psi & -\sin \psi \\
    \sin \psi & \cos \psi
    \end{array} \right]
  \end{align*}

\paragraph{Here is a specific example:}

\begin{align*}
  R( \pi/2 ) & = \left[ \begin{array}{rr}
    0 & -1 \\
    1 & 0
    \end{array} \right]
  \end{align*}

\subsubsection{Scaling}

\begin{align*}
  S( s_x, s_y ) & = \left[ \begin{array}{rr}
    s_x & 0 \\
    0 & s_y
    \end{array} \right]
  \end{align*}

\paragraph{Here is a specific example:}

\begin{align*}
  S( 2, 2 ) & = \left[ \begin{array}{rr}
    2 & 0 \\
    0 & 2
    \end{array} \right]
  \end{align*}

\subsection{Multiplication: matrix $\times$ matrix}

\paragraph{Here is the general rule:}

\begin{align*}
  A & = \left[ \begin{array}{rr}
    a_{00} & a_{01} \\
    a_{10} & a_{11}
    \end{array} \right] \\
  B & = \left[ \begin{array}{rr}
    b_{00} & b_{01} \\
    b_{10} & b_{11}
    \end{array} \right] \\
  AB & = \left[ \begin{array}{rr}
    (a_{00} b_{00} + a_{01} b_{10}) & (a_{00} b_{01} + a_{01} b_{11}) \\
    (a_{10} b_{00} + a_{11} b_{10}) & (a_{10} b_{01} + a_{11} b_{11}) \\
    \end{array} \right]
  \end{align*}

\paragraph{Here is a specific example:}

A rotation by $30^\circ$ ($\pi/6$ radians) followed by a rotation 
by $60^\circ$ ($\pi/3$ radians) produces the same result as a
single rotation by $90^\circ$ ($\pi/2$ radians).

\begin{align*}
  R(\frac{\pi}{6}) & = \left[ \begin{array}{rr}
    \frac{\sqrt{3}}{2} & -\frac{1}{2} \\
    \frac{1}{2} & \frac{\sqrt{3}}{2}
    \end{array} \right] \\
  R(\frac{\pi}{3}) & = \left[ \begin{array}{rr}
    \frac{1}{2} & -\frac{\sqrt{3}}{2} \\
    \frac{\sqrt{3}}{2} & \frac{1}{2}
    \end{array} \right] \\
  R(\frac{\pi}{2}) & = \left[ \begin{array}{rr}
    0 & -1 \\
    1 & 0
    \end{array} \right] \\
  R(\frac{\pi}{6}) \; R(\frac{\pi}{3}) & = R(\frac{\pi}{2}) 
  \end{align*}

\subsection{Multiplication: matrix $\times$ vector}

\paragraph{Here is the general rule:}

\begin{align*}
  A & = \left[ \begin{array}{rr}
    a_{00} & a_{01} \\
    a_{10} & a_{11}
    \end{array} \right] \\
  \vec{v} & = \left[ \begin{array}{r}
    v_x \\
    v_y )  
    \end{array} \right] \\
  A \vec{v} & =
    \left[ \begin{array}{rr}
      a_{00} & a_{01} \\
      a_{10} & a_{11}
      \end{array} \right]
    \left[ \begin{array}{r}
      v_x \\
      v_y
      \end{array} \right] \\
  & =
    \left[ \begin{array}{rr}
      a_{00} v_x + a_{01} v_y \\
      a_{10} v_x + a_{11} v_y
      \end{array} \right]
  \end{align*}

\paragraph{Here is a specific example:}

\subsection{Determinant}

\paragraph{Here is the general rule:}

\begin{align*}
  A & = \left[ \begin{array}{rr}
    a_{00} & a_{01} \\
    a_{10} & a_{11} 
    \end{array} \right] \\
  |A| & = a_{00} a_{11} - a_{10} a_{01}
  \end{align*}

\paragraph{Here is a specific example:}

\begin{align*}
  A & = \left[ \begin{array}{rr}
    3 & 2 \\
    6 & 8
    \end{array} \right] \\
  |A| & = 3 \cdot 8 - 2 \cdot 6 \\
      & = 24 - 12 \\
      & = 12
  \end{align*}

\subsection{Inverse}

\paragraph{Here is the general rule:}

\begin{align*}
  A & = \left[ \begin{array}{rr}
    a & b \\
    c & d 
    \end{array} \right] \\
  A^{-1} & = \frac{1}{|A|} \left[ \begin{array}{rr}
    d & -b \\
    -c & a
    \end{array} \right] \\
  A A^{-1} & = I
  \end{align*}

\paragraph{Here is a specific example:}

\begin{align*}
  A & = \left[ \begin{array}{rr}
    3 & 2 \\
    6 & 8 
    \end{array} \right] \\
  A^{-1} & = \frac{1}{|A|} \left[ \begin{array}{rr}
    8 & -2 \\
    -6 & 3
    \end{array} \right] \\
  A^{-1} & = \frac{1}{12} \left[ \begin{array}{rr}
    8 & -2 \\
    -6 & 3
    \end{array} \right] \\
  A^{-1} & = \left[ \begin{array}{rr}
    \frac{2}{3} & -\frac{1}{6} \\
    -\frac{1}{2} & \frac{1}{4}
    \end{array} \right] \\
  \left[ \begin{array}{rr}
    3 & 2 \\
    6 & 8 
    \end{array} \right] 
  \left[ \begin{array}{rr}
    \frac{2}{3} & -\frac{1}{6} \\
    -\frac{1}{2} & \frac{1}{4}
    \end{array} \right] & =
  \left[ \begin{array}{rr}
    1 & 0 \\
    0 & 1
    \end{array} \right]
  \end{align*}

\section{Matrix3x3}

\subsection{Special $3 \times 3$ matrices}

\subsubsection{Identity}

\begin{align*}
  I & = \left[ \begin{array}{rrr}
    1 & 0 & 0 \\
    0 & 1 & 0 \\
    0 & 0 & 1
    \end{array} \right]
  \end{align*}

\subsubsection{Rotation about the x-axis}

\begin{align*}
  R_x( \psi ) & = \left[ \begin{array}{rrr}
    1 & 0 & 0 \\
    0 & \cos \psi & -\sin \psi \\
    0 & \sin \psi &  \cos \psi 
    \end{array} \right]
  \end{align*}

\subsubsection{Rotation about the y-axis}

\begin{align*}
  R_y( \psi ) & = \left[ \begin{array}{rrr}
    \cos \psi & 0 & \sin \psi \\
    0 & 1 & 0 \\
    -\sin \psi & 0 & \cos \psi
    \end{array} \right]    
  \end{align*}


\subsubsection{Rotation about the z-axis}

\begin{align*}
  R_z( \psi ) & = \left[ \begin{array}{rrr}
    \cos \psi & -\sin \psi & 0 \\
    \sin \psi &  \cos \psi & 0 \\
    0 & 0 & 1
    \end{array} \right]
  \end{align*}

\subsubsection{Scaling}

\begin{align*}
  S( s_x, s_y, s_z ) & = \left[ \begin{array}{rrr}
    s_x & 0 & 0 \\
    0 & s_y & 0 \\
    0 & 0 & s_z
    \end{array} \right]
  \end{align*}

\subsection{Multiplication: matrix $\times$ matrix}

\begin{align*}
  A & = \left[ \begin{array}{rrr}
    a_{00} & a_{01} & a_{02} \\
    a_{10} & a_{11} & a_{12} \\
    a_{20} & a_{21} & a_{22} 
    \end{array} \right] \\
  B & = \left[ \begin{array}{rrr}
    b_{00} & b_{01} & b_{02} \\
    b_{10} & b_{11} & b_{12} \\
    b_{20} & b_{21} & b_{22} 
    \end{array} \right] \\
  C & = AB \\
   & = \left[ \begin{array}{rrr}
    (a_{00} b_{00} + a_{01} b_{10} + a_{02} b_{20}) &
    (a_{00} b_{01} + a_{01} b_{11} + a_{02} b_{21}) &
    (a_{00} b_{02} + a_{01} b_{12} + a_{02} b_{22}) \\
    (a_{10} b_{00} + a_{11} b_{10} + a_{12} b_{20}) &
    (a_{10} b_{01} + a_{11} b_{11} + a_{12} b_{21}) &
    (a_{10} b_{02} + a_{11} b_{12} + a_{12} b_{22}) \\
    (a_{20} b_{00} + a_{21} b_{10} + a_{22} b_{20}) &
    (a_{20} b_{01} + a_{21} b_{11} + a_{22} b_{21}) &
    (a_{20} b_{02} + a_{21} b_{12} + a_{22} b_{22}) 
    \end{array} \right]
  \end{align*}

Let $c_{ij}$ be the element in the $i^{th}$ row and $j^{th}$ column of
the $3 \times 3$ matrix $C$.
Similarly, let $a_{ij}$ and $b_{ij}$ be elements of the $3 \times 3$
matrices $A$ and $B$ whose product is $C$.

Then\ldots

\begin{align*}
  c_{ij} & = \sum_{k = 0}^2 a_{ik} \; b_{kj} \\
        & = a_{i0} \; b_{0j} + a_{i1} \; b_{1j} + a_{i2} \; b_{2j}
  \end{align*}


\subsection{Multiplication: matrix $\times$ vector}

\begin{align*}
  A & = \left[ \begin{array}{rrr}
    a_{00} & a_{01} & a_{02} \\
    a_{10} & a_{11} & a_{12} \\
    a_{20} & a_{21} & a_{22} 
    \end{array} \right] \\
  \vec{v} & = \left[ \begin{array}{r}
    v_0 \\
    v_1 \\
    v_2
    \end{array} \right] \\
  A \vec{v} & = \left[ \begin{array}{r}
    a_{00} \; v_0 + a_{01} \; v_1 + a_{02} \; v_2 \\
    a_{10} \; v_0 + a_{11} \; v_1 + a_{12} \; v_2 \\
    a_{20} \; v_0 + a_{21} \; v_1 + a_{22} \; v_2 \\
    \end{array} \right]
  \end{align*}

The element in the $i^{th}$ row 
of $A \vec{v}$ is $\sum_{k = 0}^2 a_{ik} \; v_k$.

\end{document}
